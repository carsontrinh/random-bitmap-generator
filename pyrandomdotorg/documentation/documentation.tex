% From LaTeX: It's Not Just for Academia, Part 2, Kevin O'Malley
% O'Reilly MacDevCenter, Feb 2004
%=-=-=-=-=-=-=-=-=-=-=-=-=-=-=-=-=-=-=-=-=-=-=-=-=-=-=-=-=-=-=-=-=-=
% Begin document preamble.
%=-=-=-=-=-=-=-=-=-=-=-=-=-=-=-=-=-=-=-=-=-=-=-=-=-=-=-=-=-=-=-=-=-=
\documentclass[12 pt]{report}
\usepackage{color,graphicx}
\usepackage{url}

%\setlength{\textheight}{9.45in}
%\setlength{\oddsidemargin}{-0.5in}
%\setlength{\evensidemargin}{-0.5in}
%\setlength{\topmargin}{-0.49in}
%\setlength{\footskip}{0.5in}
%\def \columnsep{0.2in}
%\def \textwidth{7.45in}

%=-=-=-=-=-=-=-=-=-=-=-=-=-=-=-=-=-=-=-=-=-=-=-=-=-=-=-=-=-=-=-=-=-=
% Start of the document (end of preamble).
%=-=-=-=-=-=-=-=-=-=-=-=-=-=-=-=-=-=-=-=-=-=-=-=-=-=-=-=-=-=-=-=-=-=
\begin{document}

\title{\vspace{-0.75in} pyRandomdotOrg - A Python Frontend for Random.org} 
\author{Sean Brewer - sbrewer@students.berry.edu}
\maketitle

%=-=-=-=-=-=-=-=-=-=-=-=-=-=-=-=-=-=-=-=-=-=-=-=-=-=-=-=-=-=-=-=-=-=
% Table of contents.
%=-=-=-=-=-=-=-=-=-=-=-=-=-=-=-=-=-=-=-=-=-=-=-=-=-=-=-=-=-=-=-=-=-=
\tableofcontents
\pagebreak
 
 
%=-=-=-=-=-=-=-=-=-=-=-=-=-=-=-=-=-=-=-=-=-=-=-=-=-=-=-=-=-=-=-=-=-=
\chapter {Introduction}
%=-=-=-=-=-=-=-=-=-=-=-=-=-=-=-=-=-=-=-=-=-=-=-=-=-=-=-=-=-=-=-=-=-=

pyRandomdotOrg is simply a Python interface to the Random.org's random
number generating service. It's fairly straightforward and easy to use. If you
need "good" random numbers and you need to use Python this is one of the 
best ways to get "good" random numbers without much hassle, as long as you
have an internet connection of course.

%=-=-=-=-=-=-=-=-=-=-=-=-=-=-=-=-=-=-=-=-=-=-=-=-=-=-=-=-=-=-=-=-=-=
\chapter{Class and Methods}
%=-=-=-=-=-=-=-=-=-=-=-=-=-=-=-=-=-=-=-=-=-=-=-=-=-=-=-=-=-=-=-=-=-=

%-------------------------------------------------------------------
\section{Information}
%-------------------------------------------------------------------

The methods and the arguments for the methods are very similar to what
is listed at http://www.random.org/clients/http/. There are a few differences
and they are described here in very good detail.

%-------------------------------------------------------------------
\section{Class}
%-------------------------------------------------------------------

There is only one class in pyRandomdotOrg, the clientlib class. It is what allows
you to connect to the Random.org random number generating service.

%_______________________________________
\subsection{clientlib}
%_______________________________________

\textbf{clientlib(clientname, emailaddr)}\\

  \begin{tabular}{ | l | l | p{10cm} | }
    \hline
    \textbf{Argument:} & \textbf{Type:} & \textbf{Information:} \\ \hline
    clientname & String & The client name (the name of your program) \\ \hline
    emailaddr & String & Your email address. This allows the Random.org admin to contact you if something goes awry with your client. \\
    \hline
  \end{tabular}


%-------------------------------------------------------------------
\section{Methods}
%-------------------------------------------------------------------

%_______________________________________
\subsection{IntegerGeneratorList}
%_______________________________________

\textbf{IntegerGeneratorList(num,nmin,nmax,col=1,base=10,rnd="new")}\\\\
This method returns a list of random integers from Random.org and returns them as a python list, i.e. [1,2,3,4,5,6].\\\\

\begin{tabular}{ | l | l | p{10cm} | }
    \hline
    \textbf{Argument:} & \textbf{Type:} & \textbf{Information:} \\ \hline
    num & Int &The number of integers requested. \\ \hline
    nmin & Int &The smallest value allowed for each integer. \\ \hline
    nmax & Int &The largest value allowed for each integer.\\ \hline
    col & Int &The number of columns in which the integers will be arranged. The integers should be read (or processed) left to right across columns. \\ \hline
    base & Int & The base that will be used to print the numbers, i.e., binary (2), octal (8), decimal (10) or hexadecimal (16). \\ \hline
    rnd & String & Determines the randomization to use to generate the strings. If new is specified, then a new randomization will created from the truly random bitstream at RANDOM.ORG. This is probably what you want in most cases. If id.identifier is specified, the identifier is used to determine the randomization in a deterministic fashion from a large pool of pregenerated random bits. Because the numbers are produced in a deterministic fashion, specifying an id basically uses RANDOM.ORG as a pseudo-random number generator. The third (date.iso-date) form is similar to the second; it allows the randomization to be based on one of the daily pregenerated files. This form must refer to one of the dates for which files exist, so it must be a day in the past. The date must be in ISO 8601  format (i.e., YYYY-MM-DD). \\
    \hline
  \end{tabular}
  

%_______________________________________
\subsection{IntegerGenerator}
%_______________________________________

\textbf{IntegerGenerator(nmin,nmax,base=10,rnd="new")}\\\\
This method returns a random integer from Random.org\\\\
\begin{tabular}{ | l | l | p{10cm} | }
    \hline
    \textbf{Argument:} & \textbf{Type:} & \textbf{Information:} \\ \hline
    nmin & Int &The smallest value allowed for each integer. \\ \hline
    nmax & Int &The largest value allowed for each integer.\\ \hline
    base & Int & The base that will be used to print the numbers, i.e., binary (2), octal (8), decimal (10) or hexadecimal (16). \\ \hline
    rnd & String & Determines the randomization to use to generate the strings. If new is specified, then a new randomization will created from the truly random bitstream at RANDOM.ORG. This is probably what you want in most cases. If id.identifier is specified, the identifier is used to determine the randomization in a deterministic fashion from a large pool of pregenerated random bits. Because the numbers are produced in a deterministic fashion, specifying an id basically uses RANDOM.ORG as a pseudo-random number generator. The third (date.iso-date) form is similar to the second; it allows the randomization to be based on one of the daily pregenerated files. This form must refer to one of the dates for which files exist, so it must be a day in the past. The date must be in ISO 8601  format (i.e., YYYY-MM-DD). \\
    \hline
  \end{tabular}

%_______________________________________
\subsection{SequenceGenerator}
%_______________________________________

\textbf{SequenceGenerator(nmin,nmax,rnd="new")}\\\\
This method returns a randomized sequence of numbers based on a particular interval. The method returns 
the list of numbers as a python list.\\\\
  \begin{tabular}{ | l | l | p{10cm} | }
    \hline
    \textbf{Argument:} & \textbf{Type:} & \textbf{Information:} \\ \hline
    nmin & Int &The lower bound of the interval (inclusive). \\ \hline
    nmax & Int &The upper bound of the interval (inclusive).\\ \hline
    rnd & String & Determines the randomization to use to generate the strings. If new is specified, then a new randomization will created from the truly random bitstream at RANDOM.ORG. This is probably what you want in most cases. If id.identifier is specified, the identifier is used to determine the randomization in a deterministic fashion from a large pool of pregenerated random bits. Because the numbers are produced in a deterministic fashion, specifying an id basically uses RANDOM.ORG as a pseudo-random number generator. The third (date.iso-date) form is similar to the second; it allows the randomization to be based on one of the daily pregenerated files. This form must refer to one of the dates for which files exist, so it must be a day in the past. The date must be in ISO 8601  format (i.e., YYYY-MM-DD). \\
    \hline
  \end{tabular}

%_______________________________________
\subsection{StringGenerator}
%_______________________________________

\textbf{StringGenerator(num,len,digits=True,upperalpha=True,}\\
\textbf{loweralpha=True,unique=True,rnd="new")}\\\\
This method gets a list of random strings from Random.org and returns them as a python list.\\\\
   \begin{tabular}{ | l | l | p{10cm} | }
    \hline
    \textbf{Argument:} & \textbf{Type:} & \textbf{Information:} \\ \hline
    num & Int & The number of strings that you want.\\ \hline
    len & Int &The length of the strings. All the strings produced will have the same length. The maximum number is 20 and the minimum is 1 \\ \hline
    digits & Boolean & Determines whether digits (0-9) are allowed to occur in the strings. \\ \hline
    upperalpha & Boolean & Determines whether uppercase alphabetic characters (A-Z) are allowed to occur in the strings. \\ \hline
    loweralpha & Boolean & Determines whether lowercase alphabetic characters (a-z) are allowed to occur in the strings. \\ \hline
    unique & Boolean &  Determines whether the strings picked should be unique (as a series of raffle tickets drawn from a hat) or not (as a series of die rolls). If unique is set to on, then there is the additional constraint that the number of strings requested (num) must be less than or equal to the number of strings that exist with the selected length and characters. \\ \hline
    rnd & String & Determines the randomization to use to generate the strings. If new is specified, then a new randomization will created from the truly random bitstream at RANDOM.ORG. This is probably what you want in most cases. If id.identifier is specified, the identifier is used to determine the randomization in a deterministic fashion from a large pool of pregenerated random bits. Because the numbers are produced in a deterministic fashion, specifying an id basically uses RANDOM.ORG as a pseudo-random number generator. The third (date.iso-date) form is similar to the second; it allows the randomization to be based on one of the daily pregenerated files. This form must refer to one of the dates for which files exist, so it must be a day in the past. The date must be in ISO 8601  format (i.e., YYYY-MM-DD). \\
    \hline
  \end{tabular}

%_______________________________________
\subsection{RandomString}
%_______________________________________

\textbf{RandomString(len,digits=True,upperalpha=True,loweralpha=True,}
\textbf{unique=True,rnd="new")}\\\\
This method gets a random string from Random.org and returns it as a string.\\\\
  \begin{tabular}{ | l | l | p{10cm} | }
    \hline
    \textbf{Argument:} & \textbf{Type:} & \textbf{Information:} \\ \hline
    len & Int &The length of the strings. All the strings produced will have the same length. The maximum number is 20 and the minimum is 1 \\ \hline
    digits & Boolean & Determines whether digits (0-9) are allowed to occur in the strings. \\ \hline
    upperalpha & Boolean & Determines whether uppercase alphabetic characters (A-Z) are allowed to occur in the strings. \\ \hline
    loweralpha & Boolean & Determines whether lowercase alphabetic characters (a-z) are allowed to occur in the strings. \\ \hline
    unique & Boolean &  Determines whether the strings picked should be unique (as a series of raffle tickets drawn from a hat) or not (as a series of die rolls). If unique is set to on, then there is the additional constraint that the number of strings requested (num) must be less than or equal to the number of strings that exist with the selected length and characters. \\ \hline
    rnd & String & Determines the randomization to use to generate the strings. If new is specified, then a new randomization will created from the truly random bitstream at RANDOM.ORG. This is probably what you want in most cases. If id.identifier is specified, the identifier is used to determine the randomization in a deterministic fashion from a large pool of pregenerated random bits. Because the numbers are produced in a deterministic fashion, specifying an id basically uses RANDOM.ORG as a pseudo-random number generator. The third (date.iso-date) form is similar to the second; it allows the randomization to be based on one of the daily pregenerated files. This form must refer to one of the dates for which files exist, so it must be a day in the past. The date must be in ISO 8601  format (i.e., YYYY-MM-DD). \\
    \hline
  \end{tabular}

%_______________________________________
\subsection{QuotaChecker}
%_______________________________________

\textbf{QuotaChecker(ipaddr=None)}\\\\
This method returns the bit quota of the given IP address. If the argument is left empty it will return the bit quota of the network
the python program is working from.\\\\
  \begin{tabular}{ | l | l | p{10cm} | }
    \hline
    \textbf{Argument:} & \textbf{Type:} & \textbf{Information:} \\ \hline
    ipaddr (optional) & String & The IP address that you want to check the quota of. If left blank your IP address will be used \\ 
    \hline
  \end{tabular}


%=-=-=-=-=-=-=-=-=-=-=-=-=-=-=-=-=-=-=-=-=-=-=-=-=-=-=-=-=-=-=-=-=-=
\chapter{Examples}
%=-=-=-=-=-=-=-=-=-=-=-=-=-=-=-=-=-=-=-=-=-=-=-=-=-=-=-=-=-=-=-=-=-=

pyRandomdotOrg is very straightforward to use. Here are some examples
as executed from the Python REPL.\\\\
\noindent
$>>>$ import pyRandomdotOrg\\
$>>>$ rnd = pyRandomdotOrg.clientlib("An Example Client","sbrewer@students.berry.edu")\\
$>>>$ print rnd.IntegerGeneratorList(5,1,1000000)\\
$[317176, 795730, 842222, 790934, 876738]$\\ 
$>>>$ print rnd.IntegerGenerator(1,1000000)\\
237629\\
$>>>$ print rnd.SequenceGenerator(1,52)\\
$[36, 46, 48, 1, 45, 50, 19, 32, 8, 28, 31, 23, 37, 2, 14, 11, 35, 49, 52, 7, 12, 40, 4, 42, 22,\\ 18, 6, 43, 10, 34, 27, 13, 25, 5, 20, 33, 44, 39, 21, 30, 16, 24, 17, 9, 41, 38, 15, 47, 29, 26, 51, 3]$\\
$>>>$ print rnd.StringGenerator(5,20)\\
$[$'2BASj44Ugk2douNKGEON', 'YyUOSZqXcQK5hqrZdtxW',\\ 'TPNiOeMrZbMR6g7u1cEQ', 'iRedoDw0h09kpThYVslB', 'INuUq2qMwPgbgUAqujzl'$]$\\
$>>>$ print rnd.RandomString(5)\\
cdcuF\\
$>>>$ print rnd.QuotaChecker()\\
996902\\
$>>>$ print rnd.QuotaChecker("6.20.28.79")\\
1000000\\
\indent
%=-=-=-=-=-=-=-=-=-=-=-=-=-=-=-=-=-=-=-=-=-=-=-=-=-=-=-=-=-=-=-=-=-=
\chapter{Conclusion/License}
%=-=-=-=-=-=-=-=-=-=-=-=-=-=-=-=-=-=-=-=-=-=-=-=-=-=-=-=-=-=-=-=-=-=

If any bugs are found in the library, do not hesitate to send me an e-mail
at sbrewer@students.berry.edu. If you know how to fix the problem and 
can provide a patch that will be even more helpful. This library is licensed 
under the GPL2, and is therefore free software. Please see the included document (GPL2.txt) 
for more details. Also, thanks to http://www.random.org/clients/http/, as I used many of the descriptions
of the arguments in this document from there. I simply couldn't word them better than that!
\end{document}
